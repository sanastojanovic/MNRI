\documentclass{beamer}
\usetheme{Madrid}
\setbeamertemplate{footline}{}

\usepackage[utf8]{inputenc}
\usepackage[T1]{fontenc}
\usepackage[serbian]{babel}
\usepackage{hyperref}

\title{BAZE PODATAKA}
\author{Jelena Žakula}
\institute{Metodika nastave računarstva A}

\begin{document}

\begin{frame}
\titlepage
\end{frame}

\begin{frame}{Rad sa tabelama i veze između njih}
\begin{itemize}
    \item Tabele su osnovna komponenta u relacionim bazama podataka. 
    \item Omogućavaju organizaciju i skladištenje podataka.
    \item Olakšavaju upravljanje i dobijanje korisnih informacija.
\end{itemize}
\end{frame}

\begin{frame}{Uređenje polja i slogova u relaciji}
\begin{itemize}
    \item Svaka tabela se sastoji od polja i slogova.
    \item Polja definišu tipove podataka, slogovi predstavljaju pojedinačne unose.
    \item Polja moraju imati jedinstven naziv i tip podataka.
    \item Svi slogovi imaju ista polja u istom redosledu.
\end{itemize}
\end{frame}

\begin{frame}{Primarni i sekundarni ključevi}
\begin{itemize}
    \item Primarni ključ - jednoznačno identifikuje svaki slog. Može biti prost (jedan atribut) ili složen (više atributa).
    \item Sekundarni ključevi (indeksi) omogućavaju bržu pretragu i sortiranje.
\end{itemize}
\end{frame}

\begin{frame}{Primer tabele sa prostim primarnim ključem}
\begin{center}
\begin{tabular}{|c|c|c|}
\hline
SifraUc & Davalac & Iznos \\
\hline
100 & Republika & 9000 \\
150 & Grad & 6000 \\
175 & Opština & 3000 \\
200 & Grad & 6000 \\
\hline
\end{tabular}

\bigskip
\textit{Napomena: Svaki učenik može primiti samo jednu stipendiju, prost primarni ključ je SifraUc.}
\end{center}
\end{frame}

\begin{frame}{Primer tabele sa složenim primarnim ključem}
\begin{center}
\begin{tabular}{|c|c|c|}
\hline
SifraUc & Davalac & Iznos \\
\hline
100 & Republika & 9000 \\
100 & Grad & 6000 \\
150 & Grad & 6000 \\
175 & Opština & 3000 \\
175 & Grad & 6000 \\
200 & Grad & 6000 \\
200 & Republika & 9000 \\
\hline
\end{tabular}

\bigskip
\textit{Napomena: Složen primarni ključ je kombinacija SifraUc i Davalac.}
\end{center}
\end{frame}

\begin{frame}{Formatiranje podataka u tabeli}
\begin{itemize}
    \item Ključna uloga za preglednost i pravilnu interpretaciju.
    \item Definisanje tipova podataka, širine kolona i formata prikaza.
    \item Olakšava čitanje, pretragu i analizu informacija.
\end{itemize}
\end{frame}

\begin{frame}{Referencijalni integritet}

Povezanost tabela ostvaruje se pomoću ključeva:
\begin{center}
\bigskip
\begin{itemize}
    \item Primarni ključ jedne tabele povezuje se sa stranim ključem druge tabele. Svaka vrednost stranog ključa mora postojati kao primarni ključ ili može biti prazna.
    \item Referencijalni integritet garantuje doslednost podataka.
\end{itemize}

\bigskip
\textit{Primer: tabele učenika i stipendija, kupaca i porudžbina.}
\end{center}
\end{frame}

\begin{frame}{Prednosti referencijalnog integriteta}
\begin{center}
\begin{itemize}
    \item Obezbeđuje logičku ispravnost baze.
    \item Automatski kontroliše unos, izmene i brisanje podataka.
    \item Omogućava kaskadne izmene i kaskadna brisanja.
    \item Olakšava spajanje tabela i formiranje složenih upita.
\end{itemize}
\end{center}
\end{frame}

\begin{frame}{Izmene veza između tabela}

Veze se mogu menjati dodavanjem, uklanjanjem ili modifikovanjem:
\begin{center}
\bigskip
\begin{itemize}
    \item Tipične veze: \textit{one-to-one}, \textit{one-to-many}, \textit{many-to-many}.
    \item Promene mogu biti: dodavanje/uklanjanje stranog ključa, promena tipa veze, redefinisanje pravila referencijalnog integriteta.
    \item Preporuka: planirati izmene unapred i analizirati postojeće podatke.
\end{itemize}
\end{center}
\end{frame}

\begin{frame}{Forme (obrasci)}

Forme omogućavaju lakši i pregledniji unos i prikaz podataka u bazi.
\begin{center}

\bigskip
\begin{itemize}
    \item Kreiranje pomoću \textit{Wizard}-a (čarobnjaka) ili ručno (\textit{Design view}).
    \item \textit{Wizard} je pogodan za početnike: automatski dodaje kontrole i izgled.
    \item Ručno kreiranje pruža maksimalnu fleksibilnost i prilagođeni vizuelni identitet.
\end{itemize}
\end{center}
\end{frame}

\begin{frame}{Unos podataka pomoću formi}

Forme služe za unos podataka u bazu bez direktnog rada sa tabelom.
\begin{center}

\bigskip
\begin{itemize}
    \item Korisnik unosi vrednosti direktno u polja tabele ili \textit{query}-a.
    \item Kontrole: \textit{text box}, \textit{combo box}, \textit{list box}, \textit{check box}.
    \item Prednost: prikaz samo relevantnih polja i validacija podataka.
\end{itemize}
\end{center}
\end{frame}


\begin{frame}{Specijalne kontrole formi}

Specijalne kontrole olakšavaju rad i navigaciju u bazi:
\begin{center}

\bigskip
\begin{itemize}
    \item \textit{List box} i \textit{combo box}: izbor vrednosti iz liste.
    \item \textit{Command buttons}: izvršavanje akcija (čuvanje, brisanje, otvaranje drugih formi).
    \item Povećavaju efikasnost i smanjuju mogućnost greške korisnika.
\end{itemize}
\end{center}
\end{frame}

\begin{frame}{Multitabelarne forme}

Omogućavaju pregled i unos podataka iz više tabela ili upita istovremeno.
\begin{center}

\bigskip
\begin{itemize}
    \item Pogodne su za prikaz povezanih podataka (npr. proizvodi i količina po magacinima).
    \item Zahtevaju pravilno povezivanje kontrola sa izvorom podataka.
    \item Često koriste \textit{subforms} za prikaz podataka iz drugih tabela ili upita.
\end{itemize}
\end{center}
\end{frame}

\begin{frame}{Pretraživanje podataka u tabeli}

Pretraživanje omogućava brzo pronalaženje željenih slogova.
\begin{center}

\bigskip
\begin{itemize}
    \item \textit{Search} funkcija: filtriranje po tekstu, brojevima ili datumu.
    \item \textit{Advanced Filter}: kombinovanje više kriterijuma.
    \item Pravilno indeksiranje ubrzava pronalaženje i optimizuje rad sa velikim bazama.
\end{itemize}
\end{center}
\end{frame}

\begin{frame}{Sortiranje, filtriranje i indeksiranje}

Sortiranje i filtriranje poboljšavaju preglednost i analizu podataka.
\begin{center}

\bigskip
\begin{itemize}
    \item \textit{Sort}: raspoređivanje slogova po rastućem ili opadajućem redosledu.
    \item \textit{Filter}: prikaz samo slogova koji ispunjavaju kriterijume.
    \item \textit{Indeksiranje}: brži pristup podacima po definisanim poljima.
    \item Kombinacija omogućava efikasno izveštavanje i vizuelizaciju.
\end{itemize}
\end{center}
\end{frame}


\begin{frame}{Osnovne SQL komande}
\begin{center}

\bigskip
\begin{itemize}
    \item Komande za definiciju podataka – kreiranje tabela i baze.
    \item Komande za manipulaciju podacima – unos, izmena i brisanje podataka.
    \item Komande za kontrolu pristupa – prava pristupa korisnika.
\end{itemize}
\end{center}
\end{frame}


\begin{frame}{Set SQL naredbi za kontrolu}

Kontrolne naredbe upravljaju pravima pristupa korisnika:
\begin{center}
\bigskip
\begin{itemize}
\item \textit{GRANT} – dodeljuje prava pristupa;
\item \textit{REVOKE} – oduzima prava pristupa.
\end{itemize}
\end{center}
\end{frame}

\begin{frame}{Set SQL naredbi za definiciju}

Definicione naredbe služe za rukovanje objektima baze:
\begin{center}
\bigskip
\begin{itemize}
\item \textit{CREATE} – kreiranje baze, tabela, pogleda i objekata;
\item \textit{ALTER} – menjanje strukture objekata;
\item \textit{DROP} – brisanje objekata ili baze.
\end{itemize}
\end{center}
\end{frame}

\begin{frame}{Set SQL naredbi za manipulaciju}

Manipulativne naredbe su najčešće korišćene:
\begin{center}

\bigskip
\begin{itemize}
\item \textit{SELECT} – čitanje podataka;
\item \textit{INSERT} – unos podataka;
\item \textit{UPDATE} – izmena podataka;
\item \textit{DELETE} – brisanje podataka.
\end{itemize}
\end{center}
\end{frame}

\begin{frame}{Kreiranje upita u bazama podataka}
\begin{center}
\begin{itemize}
    \item Pomoću čarobnjaka (\textit{Wizard}) – jednostavno, automatski dodaje kontrole i povezuje tabele.
    \item U \textit{Design} modu – precizna kontrola nad izborom tabela, polja i uslova.
\end{itemize}
\end{center}
\end{frame}

\begin{frame}{Pregled rezultata upita}

Rezultati \textit{SQL} upita prikazuju se u tabelarnom obliku:
\begin{center}
\bigskip
\begin{itemize}
    \item Omogućava analizu i interpretaciju podataka.
    \item Sortiranje, filtriranje i grupisanje olakšavaju donošenje odluka.
    \item Pomaže u otkrivanju grešaka u upitima i proveri integriteta podataka.
\end{itemize}
\end{center}
\end{frame}

\begin{frame}[fragile]{Primer \textit{SELECT} upita: svi učenici}

\textbf{Upit:}
\begin{verbatim}
SELECT *
FROM ucenik;
\end{verbatim}

\bigskip
\textbf{Rezultat:}

\begin{center}
\begin{tabular}{|c|c|c|c|c|c|c|}
\hline
id & ime & prezime & pol & datum\_rodjenja & razred & odeljenje \\
\hline
1 & Petar & Petrović & m & 2006-07-01 & 1 & 1 \\
2 & Milica & Jovanović & ž & 2006-04-03 & 1 & 1 \\
3 & Lidija & Petrović & ž & 2006-12-14 & 1 & 1 \\
4 & Petar & Milovanović & m & 2005-12-08 & 2 & 1 \\
5 & Ana & Pekić & ž & 2005-02-23 & 2 & 1 \\
\hline
\end{tabular}
\end{center}

\end{frame}

\begin{frame}[fragile]{\textit{SELECT} sa navedenim kolonama}

\textbf{Upit:}
\begin{verbatim}
SELECT id, ime, prezime, 
       pol, datum_rodjenja, 
       razred, odeljenje
FROM ucenik;
\end{verbatim}

\bigskip
\textbf{Rezultat:}

\begin{center}
\begin{tabular}{|c|c|c|c|c|c|c|}
\hline
id & ime & prezime & pol & datum\_rodjenja & razred & odeljenje \\
\hline
1 & Petar & Petrović & m & 2006-07-01 & 1 & 1 \\
2 & Milica & Jovanović & ž & 2006-04-03 & 1 & 1 \\
3 & Lidija & Petrović & ž & 2006-12-14 & 1 & 1 \\
4 & Petar & Milovanović & m & 2005-12-08 & 2 & 1 \\
5 & Ana & Pekić & ž & 2005-02-23 & 2 & 1 \\
\hline
\end{tabular}
\end{center}

\end{frame}

\begin{frame}{Pregled rezultata upita}

\begin{itemize}
    \item Rezultati se prikazuju u tabelarnom obliku.
    \item Moguće je sortirati, filtrirati i grupisati podatke.
    \item Pregled rezultata pomaže u proveri integriteta i tačnosti podataka.
    \item \textit{SELECT *} čita sve kolone iz tabele.
\end{itemize}

\end{frame}

\begin{frame}{Kreiranje multitabelarnih upita}
\begin{itemize}
    \item Omogućavaju povezivanje podataka iz više tabela.
    \item Koriste se operacije: \textit{JOIN}, \textit{INNER JOIN}, \textit{LEFT JOIN}.
    \item Važno pravilno definisati veze između tabela.
    \item Često se kombinuju sa filtriranjem, sortiranjem i grupisanjem.
\end{itemize}
\end{frame}

\begin{frame}{Izveštaji}
\begin{itemize}
    \item Organizovan prikaz podataka iz baza podataka.
    \item Služe za analizu, prezentaciju i štampu.
    \item Mogu sadržati: naslove, oznake, grafike, formatiranje.
    \item Omogućavaju pregled i vizuelizaciju podataka.
\end{itemize}
\end{frame}

\begin{frame}{Kreiranje izveštaja}
\begin{itemize}
    \item Može se koristiti alatka \textit{Report} ili \textit{Report Wizard}.
    \item \textit{Wizard} vodi kroz izbor polja, grupisanje i sortiranje.
    \item Finalni izgled izveštaja može se dodatno prilagoditi u \textit{Layout/Design}.
    \item Izveštaj se može sačuvati i naknadno menjati.
\end{itemize}
\end{frame}

\begin{frame}{Pregled izveštaja}
\begin{itemize}
    \item \textit{Report View}: privremene promene i kopiranje podataka.
    \item \textit{Layout View}: menjanje dizajna dok se podaci gledaju.
    \item \textit{Print Preview}: prikaz kako će izveštaj izgledati u štampi.
\end{itemize}
\end{frame}

\begin{frame}{Postavljanje kontrola i izračunavanja}
\begin{itemize}
    \item Kontrole omogućavaju strukturiranje i analizu podataka.
    \item Tipovi kontrola:
    \begin{itemize}
        \item \textit{Bound} (povezane) – prikaz vrednosti iz baze, npr. \textit{text box};
        \item \textit{Unbound} (nepovezane) – statički sadržaji: linije, naslovi, slike;
        \item \textit{Calculated} (izračunate) – koriste izraz umesto polja.
    \end{itemize}
\end{itemize}
\end{frame}

\begin{frame}{Postavljanje kontrola u izveštaju}
\begin{itemize}
    \item Preporučljivo prvo dodati i rasporediti povezane kontrole.
    \item Nepovezane i izračunate kontrole se dodaju naknadno.
    \item Povezivanje kontrole sa poljem:
    \begin{itemize}
        \item Prevlačenjem polja iz \textit{Field List};
        \item Direktnim unosom imena polja u \textit{ControlSource}.
    \end{itemize}
    \item Nepovezana kontrola se može naknadno povezati preko \textit{ControlSource}.
    \item Fleksibilnost omogućava kombinaciju statičkih i dinamičkih elemenata.
\end{itemize}
\end{frame}

\begin{frame}{Kreiranje multitabelarnih izveštaja}
\begin{itemize}
    \item Omogućavaju prikaz podataka iz više tabela ili upita.
    \item Veze između izvora podataka obezbeđuju ispravnu povezanost.
    \item Upotreba podizveštaja za dodatne detalje.
    \item Dizajn treba da bude pregledan:
    \begin{itemize}
        \item Raspored kolona;
        \item Boje i obrubi;
        \item Vizuelne karakteristike za brzu identifikaciju ključnih podataka.
    \end{itemize}
\end{itemize}
\end{frame}

\begin{frame}{Vizuelizacija podataka baze}
\begin{itemize}
    \item Omogućava prikaz i interpretaciju informacija iz baze.
    \item Olakšava razumevanje, analizu i donošenje odluka.
\end{itemize}
\end{frame}

\begin{frame}{Komponente za povezivanje aplikacije sa bazom}
\begin{itemize}
    \item Omogućavaju uspostavljanje i održavanje veze između aplikacije i baze.
    \item .NET komponente:
    \begin{itemize}
        \item \textit{BindingSource} – posrednik između izvora podataka i kontrola;
        \item \textit{DataTable}, \textit{DataSet}, \textit{DataView} – strukture za sinhronizaciju podataka.
    \end{itemize}
    \item Vizuelni alati (\textit{Visual Studio}) olakšavaju:
    \begin{itemize}
        \item Konfigurisanje veze;
        \item Izvršavanje upita;
        \item Upravljanje rezultatima preko \textit{data binding}-a.
    \end{itemize}
\end{itemize}
\end{frame}

\begin{frame}{Vizuelne komponente za prikaz i modifikaciju podataka}
\begin{itemize}
    \item Kontrole korisničkog interfejsa povezane sa izvorima podataka:
    \begin{itemize}
        \item \textit{DataGridView} – prikaz više zapisa u tabeli;
        \item \textit{TextBox}, \textit{ComboBox} – prikaz pojedinačnih vrednosti.
    \end{itemize}
    \item Promene u vizuelnim kontrolama automatski se propagiraju u bazu.
    \item Omogućava unos, ažuriranje i brisanje podataka bez direktnog \textit{SQL} koda.
\end{itemize}
\end{frame}

\begin{frame}{Komponente za navigaciju u aplikaciji}
\begin{itemize}
    \item Omogućavaju kretanje kroz podatke i module aplikacije.
    \item Tipične komponente:
    \begin{itemize}
        \item \textit{MenuStrip}, \textit{ToolStrip}, \textit{Toolbar}, \textit{Navigation Pane};
        \item Dugmad, kartice (\textit{tabs}), paneli.
    \end{itemize}
    \item Primer: \textit{Microsoft Access Navigation Pane} prikazuje sve objekte baze i omogućava brzi pristup.
\end{itemize}
\end{frame}

\begin{frame}{Literatura}
\begin{itemize}
    \item \href{https://petlja.org/sr-Latn-RS/kurs/4654/0}{Petlja – Uvod u baze podataka}
    \item \href{https://www.bazepodataka.matf.bg.ac.rs/}{Matematički fakultet u Beogradu – Baze podataka (nastavni materijali)}
    \item \href{https://poincare.matf.bg.ac.rs/smalkov/nastava.urbp.php}{S. Malkov – Uvod u relacione baze podataka}
    \item \href{https://poslovnainformatika.rs/access/referencijalni-integritet/}{Poslovna informatika – Referencijalni integritet u bazama podataka}
    \item \href{https://poslovnainformatika.rs/access/kreiranje-upita/?cn-reloaded=1}{Poslovna informatika – Kreiranje upita u \textit{MS Access}-u}
    \item \href{https://edukacija.rs/it/baze-podataka/sql-naredbe-setovi}{Edukacija.rs – \textit{SQL} naredbe i skupovi podataka}
    \item \href{https://petlja.org/sr-Latn-RS/kurs/4654/}{Petlja – Baze podataka (kompletan kurs)}
\end{itemize}
\end{frame}



\end{document}