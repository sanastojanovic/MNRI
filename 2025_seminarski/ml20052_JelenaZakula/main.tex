\documentclass[12pt,a4paper]{article}
\usepackage[serbian]{babel}
\usepackage[T1]{fontenc}
\usepackage[utf8]{inputenc}
\usepackage{geometry}
\usepackage[hidelinks]{hyperref}
\usepackage{setspace}
\usepackage{titlesec}
\usepackage{graphicx}
\usepackage{float}
\geometry{margin=2.5cm}
\onehalfspacing
\titleformat{\section}{\bfseries\Large}{\thesection.}{1em}{}
\titleformat{\subsection}{\bfseries\large}{\thesubsection.}{1em}{}

\begin{document}
\thispagestyle{empty}

\vspace*{6cm}

\begin{center}
{\Huge \textbf{BAZE PODATAKA}} \\[1.5cm]
{\Large Jelena Žakula}
\end{center}

\vfill

\begin{center}
{\large Matematički fakultet, Beograd, Januar 2026.}
\end{center}

\newpage

\tableofcontents
\newpage

\section{Rad sa tabelama i veze između njih}
Podsećanja radi: Tabele predstavljaju osnovnu komponentu u relacionim bazama podataka. One omogućavaju organizaciju i skladištenje podataka na način koji olakšava upravljanje podacima i dobijanje korisnih informacija. 

\subsection{Uređenje polja i slogova u relaciji}
\subsection{Formatiranje podataka u tabeli}
\subsection{Opis referencijalnog integriteta}
\subsection{Izmene veze između tabela}

\section{Forme (obrasci)}
\subsection{Kreiranje formi (obrazaca) sa i bez čarobnjaka}
\subsection{Unos podataka pomoću formi}
\subsection{Dodavanje specijalnih kontrola formi (list-boksevi, kombo-boksevi, komandna dugmad i dr.)}
\subsection{Kreiranje multitabularnih formi}

\section{Pretraživanje i sortiranje}
\subsection{Traženje informacija u tabeli}
\subsection{Sortiranje, filtriranje i indeksiranje}

\section{Upiti}

Upiti predstavljaju osnovni način komunikacije sa bazom podataka, omogućavajući korisnicima da pregledaju, dodaju, menjaju i brišu podatke prema svojim potrebama. \\

Upitni jezik SQL omogućava korisnicima da izvrše sve potrebne operacije nad bazom podataka, 
od kreiranja same baze i tabela u njoj, upisivanja, menjanja i pretraživanja podataka, 
do brisanja nekih podataka, tabela ili cele baze.

\subsection{Osnovne SQL komande (definicione, kontrolne i manipulacione)}

Komande jezika SQL se tradicionalno nazivaju upitima. Prva reč u svakom SQL upitu 
određuje vrstu akcije koju korisnik zahteva od sistema. Najčešće korišćeni upiti su:

\begin{itemize}
    \item \textbf{Upiti SELECT}, koji služe da se iz baze pročitaju željeni podaci,
    \item \textbf{Upiti INSERT}, koji služe da se u bazu upišu novi podaci,
    \item \textbf{Upiti DELETE}, koji služe da se iz baze obrišu neki podaci, i
    \item \textbf{Upiti UPDATE}, koji služe da se neki podaci u bazi ažuriraju (izmene).
\end{itemize}

Od svih pomenutih tipova upita, za nas je najznačajniji upit \textbf{SELECT}, pomoću kojeg veoma brzo možemo da dođemo do podataka \textit{(data)} koji su nam potrebni. \\[10pt]

\subsection{Kreiranje upita (sa i bez čarobnjaka)}

Kreiranje upita u SQL-u može se obaviti ručno pisanjem upita ili korišćenjem alata poznatog kao čarobnjak (wizard). 
Čarobnjak pojednostavljuje formulisanje složenih upita, posebno kada je potrebno povezati više tabela ili primeniti filtere i sortiranja. 
Za naprednije korisnike, ručno pisanje upita omogućava veću fleksibilnost i preciznost u radu sa podacima.


\subsection{Pregled rezultata upita}

Rezultati izvršenih upita prikazuju se u tabelarnom obliku, što omogućava korisnicima da brzo analiziraju i interpretiraju podatke. 
Pregled rezultata može uključivati sortiranje, filtriranje i grupisanje podataka, što pomaže u identifikaciji važnih informacija i donošenju odluka. \\

S obzirom da je upravo \textbf{SELECT} upit osnovni način za čitanje podataka iz baze, sledeći primer prirodno ilustruje ovaj proces. 
SELECT upit određuje koje kolone i iz kojih tabela želimo da dobijemo podatke, a izvršavanjem upita dobijamo tabelarni prikaz koji možemo analizirati. 
Na ovaj način, veza između pisanja upita i pregleda rezultata postaje jasna i praktično primenljiva. \\

Svaki, pa i najjednostavniji upit \textbf{SELECT} mora da sadrži sledeće elemente:
\begin{quote}
\begin{verbatim}
SELECT kolone
FROM tabele;
\end{verbatim}
\end{quote}

Ovim se sistemu nalaže da nam iz navedene tabele (ili više tabela) izdvoji sve podatke koji su dati u navedenim kolonama. 
Jezik nije osetljiv na razliku između velikih i malih slova (moguće je pisati i \texttt{select} i \texttt{from}), 
ali ćemo mi u skladu sa ustaljenom praksom, u nastavku sve ključne reči pisati velikim slovima. 
Sve beline u upitu se zanemaruju, pa samim tim nije bitno da li se upit piše u jednom ili više redova. 
Preglednosti radi, mi ćemo sve upite pisati u više redova.

Za početak ćemo uvek čitati podatke samo iz jedne tabele. Najjednostavniji slučaj je onaj u kome se želi čitanje svih podataka iz jedne tabele. 
Tada se umesto imena svih kolona može navesti samo simbol \texttt{*}. Razmotrimo sledeći primer. 
Tačka-zarez (\texttt{;}) može i ne mora da se doda na kraj upita.

Prikazati sve podatke o učenicima koji su upisani u bazi:

\begin{verbatim}
SELECT *
FROM ucenik;
\end{verbatim}

Izvršavanjem upita dobija se sledeći rezultat:

\begin{center}
\begin{tabular}{|c|l|l|c|c|c|c|}
\hline
id & ime & prezime & pol & datum\_rodjenja & razred & odeljenje \\
\hline
1 & Petar & Petrović & m & 2006-07-01 & 1 & 1 \\
2 & Milica & Jovanović & ž & 2006-04-03 & 1 & 1 \\
3 & Lidija & Petrović & ž & 2006-12-14 & 1 & 1 \\
4 & Petar & Milovanović & m & 2005-12-08 & 2 & 1 \\
5 & Ana & Pekić & ž & 2005-02-23 & 2 & 1 \\
\hline
\end{tabular}
\end{center}

Ovaj upit znači:

\begin{quote}
ODABERI sve kolone \\
IZ REDOVA tabele \texttt{ucenik}
\end{quote}

On je funkcionalno ekvivalentan sledećem upitu, ali je od njega jednostavniji za pisanje:

\begin{verbatim}
SELECT id, ime, prezime, pol, datum_rodjenja, razred, odeljenje
FROM ucenik;
\end{verbatim}

Izvršavanjem upita dobija se sledeći rezultat:

\begin{center}
\begin{tabular}{|c|l|l|c|c|c|c|}
\hline
id & ime & prezime & pol & datum\_rodjenja & razred & odeljenje \\
\hline
1 & Petar & Petrović & m & 2006-07-01 & 1 & 1 \\
2 & Milica & Jovanović & ž & 2006-04-03 & 1 & 1 \\
3 & Lidija & Petrović & ž & 2006-12-14 & 1 & 1 \\
4 & Petar & Milovanović & m & 2005-12-08 & 2 & 1 \\
5 & Ana & Pekić & ž & 2005-02-23 & 2 & 1 \\
\hline
\end{tabular}
\end{center}

Oblik upita \textbf{SELECT} koji smo upravo upoznali sadrži samo obavezne delove, pa je to najkraći mogući oblik ovog upita. 
Obradićemo različite složenije oblike upita \textbf{SELECT}, koji nam omogućavaju da od postojećih podataka odaberemo samo neke, da ih prebrojimo, da nađemo najmanji ili najveći podatak koji ispunjava neki uslov, da prikažemo jednostavne statistike po grupama podataka i slično. \\

U sistemu \textbf{SQLite Studio} se upiti pišu nakon što se klikne na kreiranu bazu \textbf{dnevnik} u prozoru 
\textbf{Databases} i potom izabere komanda menija \textbf{Tools → Open SQL Editor}. 
Kada se napiše upit, klikne se na dugme \textbf{Execute query (F9)} (plavi trouglić). 
Ukoliko se u prostoru za pisanje upita nalazi više njih, potrebno je obeležiti onaj koji želimo da pokrenemo. 
Ukoliko imamo više baza podataka, obavezno proveri da li je pored ovog dugmeta naziv baze u kojoj želiš da vršiš upite.

\begin{figure}[h]
    \centering
    \includegraphics[width=0.6\textwidth]{slika1.png} 
\end{figure}

\begin{figure}[h]
    \centering
    \includegraphics[width=0.6\textwidth]{slika2.png} 
\end{figure}

Često će nam kod upita biti potrebno da znamo i tačne nazive kolona. 
To možemo da vidimo za svaku tabelu pojedinačno tako što kliknemo na nju u prozoru \textbf{Databases}, 
pa se onda pojavi opis strukture tabele koji sadrži spisak svih kolona.

\begin{figure}[H]
    \centering
    \includegraphics[width=0.6\textwidth]{slika3.png} 
\end{figure}

Spisak kolona možemo da vidimo i kada izvršimo osnovni \textbf{SELECT} upit:

\begin{figure}[H]
    \centering
    \includegraphics[width=0.6\textwidth]{slika4.png} 
\end{figure}

\subsection{Kreiranje multitabularnih upita}

Multitabularni upiti (multi-table queries) omogućavaju korisnicima da povežu podatke iz više tabela koristeći operacije kao što su \textbf{JOIN}, \textbf{INNER JOIN} i \textbf{LEFT JOIN}. 
Ovi upiti su posebno korisni kada je potrebno kombinovati informacije iz različitih delova baze podataka kako bi se dobio potpuniji uvid u podatke.


\section{Izveštaji}
\subsection{Kreiranje izveštaja (sa i bez čarobnjaka)}
\subsection{Pregled izveštaja}
\subsection{Postavljanje kontrola i izračunavanja u izveštajima}
\subsection{Kreiranje multitabularnih izveštaja}

\section{Vizuelizacija podataka iz baze}
\subsection{Komponente za povezivanje Windows-aplikacije sa bazom podataka}
\subsection{Vizuelne komponente za prikazivanje i modifikaciju podataka baze}
\subsection{Komponente za navigaciju}

\end{document}
