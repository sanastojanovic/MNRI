\documentclass[12pt,a4paper]{article}
\usepackage[serbian]{babel}
\usepackage[T1]{fontenc}
\usepackage[utf8]{inputenc}
\usepackage{geometry}
\usepackage[hidelinks]{hyperref}
\usepackage{setspace}
\usepackage{titlesec}
\usepackage{graphicx}
\usepackage{float}
\geometry{margin=2.5cm}
\onehalfspacing
\titleformat{\section}{\bfseries\Large}{\thesection.}{1em}{}
\titleformat{\subsection}{\bfseries\large}{\thesubsection.}{1em}{}

\begin{document}
\thispagestyle{empty}

\vspace*{6cm}

\begin{center}
{\Huge \textbf{BAZE PODATAKA}} \\[1.5cm]
{\Large Jelena Žakula}
\end{center}

\vfill

\noindent
\begin{minipage}[t]{0.5\textwidth}
{\large Metodika nastave računarstva A, 2025/2026}
\end{minipage}
\begin{minipage}[t]{0.5\textwidth}
\begin{flushright}
{\large Sana Stojanović Đurđević}
\end{flushright}
\end{minipage}

\newpage

\tableofcontents
\newpage

\section{Rad sa tabelama i veze između njih}
Tabele predstavljaju osnovnu komponentu u relacionim bazama podataka. One omogućavaju organizaciju i skladištenje podataka na način koji olakšava upravljanje podacima i dobijanje korisnih informacija. 

\subsection{Uređenje polja i slogova u relaciji}

U relacionim bazama podataka, svaka tabela se sastoji od polja i slogova. 
Polja definišu tipove podataka koje tabela može da sadrži, dok slogovi predstavljaju pojedinačne unose podataka. 
Pravilno uređenje polja uključuje odabir odgovarajućih tipova podataka, kao i postavljanje ključeva i indeksa radi brže i efikasnije pretrage. 
Slogovi se unose tako da budu konzistentni i u skladu sa definisanim tipovima polja. \\

Svi slogovi sadrže ista polja, koja se u svakom redu pojavljuju u istom redosledu, dok svako polje ima jedinstven naziv i unapred definisan tip podataka, kao što su numerički, tekstualni, logički ili datumski tip. 
Podaci unutar jednog polja moraju pripadati istom domenu, odnosno skupu dozvoljenih vrednosti za taj atribut.

Jedno od osnovnih pravila relacionog modela jeste da se u tabeli ne mogu nalaziti dva potpuno ista sloga. 
Ova jedinstvenost se ostvaruje uvođenjem primarnog ključa, koji predstavlja atribut ili skup atributa čije vrednosti jednoznačno identifikuju svaki slog u tabeli. 
Primarni ključ može biti prost, ukoliko ga čini jedan atribut, ili složen, kada se sastoji od više atributa zajedno. \\

U praksi se kao primarni ključ često koriste jedinstveni identifikatori, poput broja indeksa studenta, registracije vozila itd. \\

U slučajevima kada ne postoji prirodan atribut koji ispunjava uslove primarnog ključa, sistemi za upravljanje bazama podataka omogućavaju korišćenje veštačkog ključa, najčešće polja tipa \textit{AutoIncrement}, koje se automatski uvećava za svaki novi slog.

Pored primarnog ključa, u tabelama se mogu koristiti i sekundarni ključevi, odnosno indeksi. Indeksi ne moraju imati jedinstvene vrednosti, ali omogućavaju brže pretraživanje i sortiranje podataka, slično indeksima u knjigama. \\

Pretraga podataka može se vršiti sekvencijalno, čitanjem svakog sloga, ili korišćenjem indeksa, gde se vrednosti čuvaju u dodatnoj strukturi radi efikasnijeg pristupa podacima. \\[10pt]

U sledećem primeru tabele, učenik sa atributom SifraUc može da prima samo jednu stipendiju, tako da ovo polje može da bude primarni ključ, i to prost ključ.

\begin{table}[h]
\centering
\begin{tabular}{|c|l|c|}
\hline
\textbf{SifraUc} & \textbf{Davalac} & \textbf{Iznos} \\
\hline
100 & Republika & 9000 \\
150 & Grad      & 6000 \\
175 & Opština   & 3000 \\
200 & Grad      & 6000 \\
\hline
\end{tabular}
\caption{Primer tabele sa prostim primarnim ključem}
\end{table}

U drugom primeru učenik može da prima više različitih stipendija (ali samo jednu od jednog davaoca stipendije), tako da u ovom slučaju polja SifraUc i Davalac zajedno mogu da čine složeni primarni ključ (učenik sa šifrom 100 može dobiti samo jednu stipendiju od Republike, ne mogu se u tabeli naći dve iste takve vrednosti).

\begin{table}[h]
\centering
\begin{tabular}{|c|l|c|}
\hline
\textbf{SifraUc} & \textbf{Davalac} & \textbf{Iznos} \\
\hline
100 & Republika & 9000 \\
100 & Grad      & 6000 \\
150 & Grad      & 6000 \\
175 & Opština   & 3000 \\
175 & Grad      & 6000 \\
200 & Grad      & 6000 \\
200 & Republika & 9000 \\
\hline
\end{tabular}
\caption{Primer tabele sa složenim primarnim ključem}
\end{table}

\subsection{Formatiranje podataka u tabeli}

Formatiranje podataka u tabeli ima ključnu ulogu u obezbeđivanju preglednosti, konzistentnosti i pravilne interpretacije informacija. 

Ono obuhvata definisanje tipova podataka za polja, postavljanje širine kolona, kao i izbor odgovarajućeg formata prikaza za numeričke vrednosti, datume i tekstualne podatke. Pravilno formatirana tabela omogućava lakše čitanje podataka i smanjuje mogućnost pogrešne interpretacije.

Pored osnovnog izgleda, formatiranje uključuje i primenu pravila prikaza, kao što su ograničenja unosa, podrazumevane vrednosti i maskiranje podataka. 

Ovakva pravila doprinose tačnosti unosa i obezbeđuju da se podaci unose u skladu sa unapred definisanim standardima. Dobar dizajn i formatiranje tabele olakšavaju rad sa podacima, unapređuju kvalitet baze i predstavljaju osnovu za efikasnu obradu, pretragu i analizu informacija u relacionim bazama podataka.


\subsection{Opis referencijalnog integriteta}

U relacionim bazama podataka tabele se često ne posmatraju izolovano, već su međusobno povezane kako bi se izbeglo dupliranje podataka i obezbedila njihova logička povezanost. Takve veze između tabela ostvaruju se pomoću ključeva, pri čemu se najčešće povezuje primarni ključ jedne tabele sa stranim ključem druge tabele. Da bi ove veze bile ispravne i pouzdane, neophodno je obezbediti referencijalni integritet. \\

Referencijalni integritet predstavlja skup pravila koja garantuju doslednost podataka između povezanih tabela. Njegova osnovna uloga je da obezbedi da svaka vrednost stranog ključa u jednoj tabeli odgovara postojećoj vrednosti primarnog ključa u povezanoj tabeli, ili da bude prazna ukoliko je to dozvoljeno. 

Primena referencijalnog integriteta posebno je važna prilikom povezivanja tabela koje opisuju različite aspekte istog sistema, kao što su npr. tabele učenika i stipendija, kupaca i porudžbina ili proizvoda i magacina. 

Kada se jednom uspostavi veza između tabela, sistem za upravljanje bazom podataka automatski kontroliše operacije unosa, izmene i brisanja podataka, čime se čuva logička ispravnost baze. \\

Referencijalni integritet takođe omogućava definisanje dodatnih pravila ponašanja, kao što su kaskadne izmene i kaskadna brisanja. Ova pravila određuju kako će se promene u jednoj tabeli odraziti na povezane zapise u drugim tabelama. Na taj način se obezbeđuje stabilnost baze podataka i olakšava rad sa povezanim tabelama, što predstavlja osnovu za efikasno spajanje tabela, formiranje upita i izradu složenijih struktura u relacionim bazama podataka.


\subsection{Izmene veze između tabela}

Veze između tabela mogu se menjati dodavanjem, uklanjanjem ili modifikovanjem veza između polja. 
Tipične veze su \textit{one-to-one}, \textit{one-to-many} i \textit{many-to-many}. Do izmene veze može doći usled promene zahteva sistema, proširenja baze podataka ili potrebe za drugačijom organizacijom podataka. Takve izmene mogu uključivati dodavanje ili uklanjanje stranog ključa, promenu tipa veze, kao i redefinisanje pravila referencijalnog integriteta. \\

U praksi se preporučuje da se izmene veza između tabela planiraju unapred i sprovode uz prethodnu analizu postojećih podataka. Često je neophodno izvršiti migraciju podataka ili privremeno prilagoditi strukturu baze kako bi se sačuvala ispravnost i stabilnost sistema.

\section{Forme (obrasci)}

\subsection{Kreiranje formi (obrazaca) sa i bez čarobnjaka}

Forme predstavljaju jedan od najvažnijih alata u radu sa bazama podataka, jer omogućavaju lakši i pregledniji unos i prikaz podataka. Mogu biti kreirane na više načina: korišćenjem \textit{Wizard} funkcije, tj pomoću čarobnjaka, koja vodi korisnika kroz proces kreiranja, ili ručno, u \textit{Design view}, gde se sve postavke i raspored kontrola podešavaju samostalno.

\textit{Wizard} je pogodan za početnike jer automatski dodaje standardne kontrole i postavlja osnovni izgled forme. U kreiranju forme bez \textit{Wizard-a}, korisnik ima potpunu kontrolu nad rasporedom polja, vrstom kontrola i vizuelnim izgledom forme. Ovo je naročito korisno kada je potrebno napraviti prilagođenu formu koja sadrži složene kontrole ili kada se želi postići specifičan vizuelni identitet baze podataka. Forme kreirane ovako zahtevaju detaljnije planiranje, ali pružaju maksimalnu fleksibilnost.

\subsection{Unos podataka pomoću formi}

Forme služe prvenstveno za unos podataka u bazu. Pomoću povezanih kontrola, korisnik može unositi vrednosti direktno u polja iz osnovne tabele ili \textit{query-a}. Svaka kontrola može biti tipa \textit{text box}, \textit{combo box}, \textit{list box} ili \textit{check box}, zavisno od vrste podataka koji se unose. Prednost formi je u tome što korisnik ne vidi kompletnu strukturu tabele, već samo relevantna polja, što smanjuje mogućnost greške.

Takođe, forme omogućavaju validaciju podataka i korišćenje izraza za izračunavanje vrednosti, što olakšava unos i održavanje integriteta baze podataka.

\subsection{Dodavanje specijalnih kontrola formi}

Pored standardnih tekstualnih i numeričkih polja, forme mogu sadržati specijalne kontrole koje olakšavaju rad sa podacima i navigaciju. \textit{List boxes} i \textit{combo boxes} omogućavaju izbor vrednosti iz unapred definisanih lista, dok \textit{command buttons} mogu izvršavati akcije kao što su čuvanje, brisanje ili otvaranje drugih formi. Korišćenje ovih kontrola povećava efikasnost i smanjuje mogućnost greške korisnika. 

\subsection{Kreiranje multitabularnih formi}

Multitabularne forme ili višestruke tabele unutar jedne forme omogućavaju pregled i unos podataka iz više tabela ili upita istovremeno. Ovakve forme su korisne za prikaz povezanih podataka, na primer prikaz proizvoda i njihove trenutne količine po magacinima.

Kreiranje ovakvih formi zahteva pravilno povezivanje kontrola sa izvorom podataka i često korišćenje \textit{subforms} za prikaz podataka iz drugih tabela ili upita.

\section{Pretraživanje i sortiranje}

\subsection{Traženje informacija u tabeli}

Pretraživanje podataka predstavlja osnovnu funkcionalnost u radu sa bazom podataka. Kroz forme, korisnik može brzo pronaći željene slogove prema različitim kriterijumima. \textit{Search} funkcija omogućava filtriranje podataka po tekstu, brojevima ili datumu, dok \textit{Advanced Filter} omogućava kombinovanje više kriterijuma. Pravilno postavljeni kriterijumi i indeksiranje tabela omogućavaju brže pronalaženje podataka i optimizuju rad sa velikim bazama.

\subsection{Sortiranje, filtriranje i indeksiranje}

Sortiranje i filtriranje podataka poboljšavaju preglednost i analizu informacija. \textit{Sort} opcija omogućava raspoređivanje slogova po rastućem ili opadajućem redosledu, dok \textit{Filter} omogućava prikaz samo onih slogova koji ispunjavaju određene kriterijume. Indeksiranje tabela dodatno ubrzava ove operacije, jer omogućava brži pristup podacima po definisanim poljima. Kombinacija sortiranih, filtriranih i indeksiranih tabela je posebno korisna prilikom izveštavanja i vizuelizacije podataka u formama i izveštajima, jer omogućava korisniku da se fokusira na relevantne informacije i olakšava donošenje poslovnih odluka.

\section{Upiti}

Upiti predstavljaju osnovni način komunikacije sa bazom podataka, omogućavajući korisnicima da pregledaju, dodaju, menjaju i brišu podatke prema svojim potrebama.

Upitni jezik SQL omogućava korisnicima da izvrše sve potrebne operacije nad bazom podataka, od kreiranja same baze i tabela u njoj, upisivanja, menjanja i pretraživanja podataka, do brisanja nekih podataka, tabela ili cele baze.

\subsection{Osnovne SQL komande (definicione, kontrolne i manipulacione)}

Komande jezika \textit{SQL} se tradicionalno nazivaju upitima. Prva reč u svakom \textit{SQL} upitu određuje vrstu akcije koju korisnik zahteva od sistema. Na osnovu te akcije, \textit{SQL} komande se logički dele u tri osnovna seta: 
komande za definiciju podataka, komande za manipulaciju podacima i komande za kontrolu pristupa. \\

Ova podela se u praksi ne ističe posebno, jer se sve naredbe koriste kroz upite na sličan način, ali je veoma značajna za razumevanje njihove namene i uloge u radu sa bazama podataka. \\

Najčešće korišćeni \textit{SQL} upiti su:
\begin{itemize}
    \item \textbf{\textit{SELECT}}, koji služe za čitanje podataka iz baze,
    \item \textbf{\textit{INSERT}}, koji služe za unos novih podataka u bazu,
    \item \textbf{\textit{DELETE}}, koji služe za brisanje postojećih podataka,
    \item \textbf{\textit{UPDATE}}, koji služe za izmenu postojećih podataka u bazi.
\end{itemize}

\subsubsection{Set SQL naredbi za kontrolu}

Naredbe za kontrolu koriste se za upravljanje pravima pristupa na serveru. 
One određuju koji korisnici mogu da izvršavaju određene operacije nad bazama i objektima u njima. 
U ovaj set spadaju naredbe \textit{GRANT} i \textit{REVOKE}.

\subsubsection{Set SQL naredbi za definiciju}

Set definicionih naredbi koristi se za rukovanje objektima na serveru. U ovaj set spadaju naredbe \textit{CREATE}, \textit{ALTER} i \textit{DROP}. Njima se kreiraju, menjaju ili brišu baze podataka, tabele, pogledi, uskladištene procedure i drugi objekti. 

Pošto se ovim naredbama upravlja strukturom baze, 
one su najčešće dostupne samo administratorima servera. 
Korisnici obično imaju prava da rade unutar svoje baze, 
ali ne i da brišu baze ili upravljaju drugim bazama na serveru. 

\subsubsection{Set SQL naredbi za manipulaciju}

Naredbe za manipulaciju podacima predstavljaju srž jezika \textit{SQL}. One se koriste najčešće i u najvećoj količini, jer omogućavaju svakodnevni rad sa podacima. U ovaj set spadaju naredbe \textit{SELECT}, \textit{INSERT}, \textit{UPDATE} i \textit{DELETE}.

\subsection{Kreiranje upita (sa i bez čarobnjaka)}

Upiti u sistemu za upravljanje bazama podataka mogu se kreirati na više načina, najčešće pomoću čarobnjaka ili u \textit{design} modu. Čarobnjak pojednostavljuje formulisanje složenih upita, posebno kada je potrebno povezati više tabela ili primeniti filtere i sortiranja.

Kreiranje upita u \textit{design} modu omogućava preciznu kontrolu nad izborom tabela, polja i uslova, ali zahteva poznavanje strukture baze podataka, kao i relacija između tabela. Pre samog kreiranja upita, neophodno je jasno definisati zadatak, odnosno utvrditi koje podatke je potrebno prikazati. 

\subsection{Pregled rezultata upita}

Rezultati izvršenih upita prikazuju se u tabelarnom obliku, što omogućava korisnicima da brzo analiziraju i interpretiraju podatke. Pregled rezultata može uključivati sortiranje, filtriranje i grupisanje podataka, što pomaže u identifikaciji važnih informacija i donošenju odluka.

Tokom pregleda rezultata moguće je uočiti greške u logici upita, kao što su pogrešno definisani uslovi, neispravne veze između tabela ili nedostajući podaci. Zbog toga pregled rezultata ima i kontrolnu ulogu, jer pomaže u proveri integriteta i tačnosti podataka. \\

S obzirom da je upravo \textit{SELECT} upit osnovni način za čitanje podataka iz baze, sledeći primer prirodno ilustruje ovaj proces. Svaki \textit{SELECT} mora da sadrži sledeće elemente:
\begin{quote}
\textit{SELECT} kolone \\
\textit{FROM} tabele;
\end{quote}

Ovim se sistemu nalaže da nam iz navedene tabele (ili više tabela) izdvoji sve podatke koji su dati u navedenim kolonama. Za početak ćemo uvek čitati podatke samo iz jedne tabele. Najjednostavniji slučaj je onaj u kome se želi čitanje svih podataka iz jedne tabele. Tada se umesto imena svih kolona može navesti samo simbol \texttt{*}.

Primer: Prikazati sve podatke o učenicima koji su upisani u bazi:

\begin{verbatim}
SELECT *
FROM ucenik;
\end{verbatim}

Izvršavanjem upita dobija se sledeći rezultat:

\begin{center}
\begin{tabular}{|c|l|l|c|c|c|c|}
\hline
id & ime & prezime & pol & datum\_rodjenja & razred & odeljenje \\
\hline
1 & Petar & Petrović & m & 2006-07-01 & 1 & 1 \\
2 & Milica & Jovanović & ž & 2006-04-03 & 1 & 1 \\
3 & Lidija & Petrović & ž & 2006-12-14 & 1 & 1 \\
4 & Petar & Milovanović & m & 2005-12-08 & 2 & 1 \\
5 & Ana & Pekić & ž & 2005-02-23 & 2 & 1 \\
\hline
\end{tabular}
\end{center}

Ovaj upit znači:

\begin{quote}
ODABERI sve kolone \\
IZ REDOVA tabele \texttt{ucenik}
\end{quote}

On je funkcionalno ekvivalentan sledećem upitu, ali je od njega jednostavniji za pisanje:

\begin{verbatim}
SELECT id, ime, prezime, pol, datum_rodjenja, razred, odeljenje
FROM ucenik;
\end{verbatim}

Izvršavanjem upita dobija se sledeći rezultat:

\begin{center}
\begin{tabular}{|c|l|l|c|c|c|c|}
\hline
id & ime & prezime & pol & datum\_rodjenja & razred & odeljenje \\
\hline
1 & Petar & Petrović & m & 2006-07-01 & 1 & 1 \\
2 & Milica & Jovanović & ž & 2006-04-03 & 1 & 1 \\
3 & Lidija & Petrović & ž & 2006-12-14 & 1 & 1 \\
4 & Petar & Milovanović & m & 2005-12-08 & 2 & 1 \\
5 & Ana & Pekić & ž & 2005-02-23 & 2 & 1 \\
\hline
\end{tabular}
\end{center} 

Obradićemo različite složenije oblike upita \textit{SELECT}, koji nam omogućavaju da od postojećih podataka odaberemo samo neke, da ih prebrojimo, da nađemo najmanji ili najveći podatak koji ispunjava neki uslov, i slično. \\

U sistemu \textit{SQLite Studio} se upiti pišu nakon što se klikne na kreiranu bazu \textbf{dnevnik} u prozoru 
\textit{Databases} i potom izabere komanda menija \textit{Tools → Open SQL Editor}. Kada se napiše upit, klikne se na dugme \textit{Execute query (F9)} (plavi trouglić). Kako je prikazano na slici. 

\begin{figure}[H]
    \centering
    \includegraphics[width=0.6\textwidth]{slika1.png} 
\end{figure}

Ukoliko se u prostoru za pisanje upita nalazi više njih, potrebno je obeležiti onaj koji želimo da pokrenemo. 
Ukoliko imamo više baza podataka, obavezno proveri da li je pored ovog dugmeta naziv baze u kojoj želiš da vršiš upite. \\

\begin{figure}[H]
    \centering
    \includegraphics[width=0.6\textwidth]{slika2.png} 
\end{figure}

Često će nam kod upita biti potrebno da znamo i tačne nazive kolona. To možemo da vidimo za svaku tabelu pojedinačno tako što kliknemo na nju u prozoru \textit{Databases}, pa se onda pojavi opis strukture tabele koji sadrži spisak svih kolona.

\begin{figure}[H]
    \centering
    \includegraphics[width=0.6\textwidth]{slika3.png} 
\end{figure}

Spisak kolona možemo da vidimo i kada izvršimo osnovni \textit{SELECT} upit:

\begin{figure}[H]
    \centering
    \includegraphics[width=0.6\textwidth]{slika4.png} 
\end{figure}

\subsection{Kreiranje multitabelarnih upita}

Multitabularni upiti omogućavaju korisnicima da povežu podatke iz više tabela koristeći operacije kao što su \textit{JOIN}, \textit{INNER JOIN} i \textit{LEFT JOIN}. 
Ovi upiti su posebno korisni kada je potrebno kombinovati informacije iz različitih delova baze podataka kako bi se dobio potpuniji uvid u podatke.

Prilikom kreiranja multitabelarnih upita važno je voditi računa o strukturi baze i pravilnom definisanju veza između tabela. Neispravno povezivanje može dovesti do pojave dupliranih slogova ili do prevelikog broja rezultata, što otežava analizu podataka. Zbog toga se multitabelarni upiti često kombinuju sa dodatnim uslovima filtriranja, sortiranja i grupisanja, čime se dobijaju pregledni i smisleni rezultati koji verno odražavaju stvarne odnose između podataka u bazi.

\section{Izveštaji}

Izveštaji predstavljaju organizovan prikaz podataka iz baza podataka, namenjen analizi, prezentaciji i štampi. 
Oni se sastoje od informacija koje se preuzimaju iz tabela ili upita, kao i dodatnih elemenata poput oznaka, naslova, grafika i formata koji se definišu prilikom dizajna izveštaja.

Izveštaji omogućavaju pregled podataka na pregledan i čitljiv način, često uključujući sortiranje, grupisanje, filtriranje i agregacije. 

Pored analitičke funkcije, izveštaji služe i kao dokumentacija i alat za donošenje odluka, jer omogućavaju vizuelizaciju važnih podataka i njihovih odnosa.

\subsection{Kreiranje izveštaja}

Kreiranje izveštaja može se izvršiti na više načina. Jedan od najbržih je korišćenje alatke \textit{Report} koja automatski generiše izveštaj na osnovu izabranog izvora podataka. Ova metoda omogućava brzo pregledanje osnovnih podataka, dok finalni izgled izveštaja može biti dodatno uređivan u prikazu \textit{Layout} ili \textit{Design}, kako bi se postigla bolja preglednost i estetski efekat.

\subsubsection{Kreiranje izveštaja pomoću čarobnjaka}

Čarobnjak za izveštaje (\textit{Report Wizard}) omogućava korisniku da precizno odredi koja polja će biti uključena u izveštaj, kako će podaci biti grupisani i sortirani, kao i da koristi polja iz više tabela ili upita, pod uslovom da su prethodno definisane relacije između tabela.  

\begin{enumerate}
    \item Na kartici \textit{Create}, u grupi \textit{Reports}, izaberite opciju \textit{Report Wizard}.
    \item Pratite uputstva čarobnjaka, birajući polja, način grupisanja i sortiranja podataka.
    \item Na poslednjoj stranici kliknite \textit{Finish} da biste generisali izveštaj.
\end{enumerate}

Izveštaj se može sačuvati i naknadno menjati, čime se omogućava stalno ažuriranje podataka iz izvora zapisa i prilagođavanje izgleda potrebama korisnika.

\subsection{Pregled izveštaja}

Postoji više načina za prikaz izveštaja. Izbor metoda zavisi od toga šta želite da uradite sa izveštajem i njegovim podacima:  

\begin{itemize}
    \item Ako je cilj napraviti privremene promene u tome koji podaci se pojavljuju u izveštaju pre štampanja, ili ako želite da kopirate podatke iz izveštaja u drugi dokument, koristi se \textit{Report View}.
    \item Ako želite mogućnost da promenite dizajn izveštaja dok posmatrate podatke, koristi se \textit{Layout View}.
    \item Ako je potrebno samo videti kako će izveštaj izgledati kada se odštampa, koristi se \textit{Print Preview}.
\end{itemize}


\subsection{Postavljanje kontrola i izračunavanja u izveštajima}

Postavljanje kontrola i izračunavanja u izveštajima ima ključnu ulogu u jasnoj i tačnoj prezentaciji podataka iz baze. Izveštaji se ne koriste samo za prost prikaz podataka, već i za njihovu analizu, sumiranje i tumačenje. Kontrole u izveštajima (tekstualna polja, labele, slike, linije i dr.) omogućavaju strukturiranje sadržaja, isticanje važnih informacija i logičko razdvajanje podataka po sekcijama, kao što su zaglavlje, telo izveštaja i podnožje.
Kontrole se mogu svrstati u tri osnovna tipa: povezane (\textit{bound}), nepovezane (\textit{unbound}) i izračunate (\textit{calculated}).  

\subsubsection{Povezane kontrole}

Povezana kontrola je kontrola čiji je \textit{data source} polje u tabeli ili upitu. Ove kontrole se koriste za prikaz vrednosti iz baze podataka, uključujući tekst, brojeve, datume, vrednosti \textit{Yes/No}, slike i grafike. Najčešći tip povezane kontrole je \textit{text box}. 

\subsubsection{Nepovezane kontrole}

Nepovezana kontrola nema izvor podataka i koristi se za prikaz statičkih informacija, linija, pravougaonika, naslova ili slika. Na primer, oznaka koja prikazuje naslov izveštaja je nepovezana kontrola. One omogućavaju vizuelno obogaćivanje izveštaja i organizovanje informacija nezavisno od baze podataka.  

\subsubsection{Izračunate kontrole}

Izračunata kontrola koristi izraz umesto polja kao \textit{data source}. Izraz može sadržati operatore (npr. =, +, *, /), imena kontrola, polja iz izvora zapisa ili konstantne vrednosti. Na primer, izraz:  

\begin{verbatim}
= [Cena po jedinici] * 0,75
\end{verbatim}

\subsubsection{Postavljanje kontrola u izveštaju}

Prilikom kreiranja izveštaja, preporučljivo je prvo dodati i rasporediti sve povezane kontrole, jer one čine osnovu većine izveštaja. Nepovezane i izračunate kontrole se potom dodaju kako bi se dopunio dizajn i omogućilo izračunavanje dodatnih vrednosti. \\ 

Povezivanje kontrole sa poljem može se ostvariti prevlačenjem polja iz okna \textit{Field List} u izveštaj. Okno \textit{Field List} prikazuje polja iz izvora zapisa izveštaja (tabele ili upita). Alternativno, ime polja može se uneti direktno u kontrolu ili u svojstvo \textit{ControlSource} u listi svojstava kontrole. Lista svojstava definiše karakteristike kontrole, uključujući ime, izvor podataka i format.  

Nepovezana kontrola se može povezati sa poljem naknadno, tako što se svojstvo \textit{ControlSource} postavi na ime polja. Ova fleksibilnost omogućava dizajniranje izveštaja koji kombinuju statičke i dinamičke elemente, prilagođene potrebama krajnjih korisnika.  

\subsection{Kreiranje multitabelarnih izveštaja}

Multitabularni izveštaji omogućavaju prikaz podataka iz više tabela ili upita u jednom izveštaju, čime se olakšava analiza i upoređivanje informacija iz različitih izvora. Ovakvi izveštaji se kreiraju tako što se kao izvori zapisa koriste tabele ili prethodno definisani upiti, a između njih se uspostavljaju relacije kako bi se podaci ispravno povezali. Upotreba podizveštaja omogućava uključivanje dodatnih detalja iz povezanih tabela unutar glavnog izveštaja, čime se dobija pregledan i informativan izveštaj.  \\

Dizajn multitabularnog izveštaja treba da bude jasan i pregledan: raspored kolona, boje, obrubi i druge vizuelne karakteristike pomažu korisniku da brzo identifikuje ključne podatke. Ovakvi izveštaji su posebno korisni za praćenje višedimenzionalnih informacija, kao što su lageri po magacinima, prodaja po proizvodima ili finansijski pokazatelji po odeljenjima.

\section{Vizuelizacija podataka baze}

Vizuelizacija podataka baze predstavlja prikaz i interpretaciju informacija iz baze na način koji olakšava razumevanje, analizu i donošenje odluka.

\subsection{Komponente za povezivanje \textit{Windows}-aplikacije sa bazom podataka}

Povezivanje \textit{Windows}-aplikacije sa bazom podataka obuhvata komponente i mehanizme koji omogućavaju uspostavljanje i održavanje veze između aplikacije i izvora podataka.

U .NET okruženju, vizuelne komponente se često oslanjaju na strukture kao što su \textit{BindingSource}, \textit{DataTable}, \textit{DataSet} i \textit{DataView}, koje omogućavaju automatsku sinhronizaciju podataka između baze i elemenata korisničkog \textit{interfejsa}. \\ \textit{BindingSource} predstavlja posrednika koji povezuje izvor podataka sa više kontrola, omogućavajući dvosmerno ažuriranje i obaveštavanje o promenama.
Sa druge strane, \textit{Visual Studio} i druge razvojne platforme nude alate koji olakšavaju konfigurisanje veze, izvršavanje upita i upravljanje rezultatima direktno u kodu ili kroz vizuelne komponente koje podržavaju \textit{data binding}.

\subsection{Vizuelne komponente za prikazivanje i modifikaciju podataka baze}

Vizuelne komponente predstavljaju kontrolne elemente korisničkog interfejsa koje omogućavaju prikaz i modifikaciju podataka iz baze. Najčešće se koriste tabele, liste, obrasci, padajuće liste i grafikoni koji su povezani sa izvorima podataka putem mehanizama \textit{data binding}. Na primer, komponenta \textit{DataGridView} omogućava prikaz višestrukih zapisa baze u tabelarnom obliku, dok kontrole poput \textit{TextBox} ili \textit{ComboBox} služe za prikaz pojedinačnih vrednosti ili izbor iz skupa podataka. 

Ove komponente omogućavaju ne samo prikaz podataka, već i njihovu modifikaciju, jer se promene u vizuelnim kontrolama mogu automatski propagirati nazad u izvor podataka. To znači da korisnik može unositi, ažurirati ili brisati podatke direktno, bez potrebe za direktnim pisanjem \textit{SQL} upita. Vizuelni alati za razvoj aplikacija kao što je \textit{Visual Studio} dodatno podržavaju automatsko generisanje i povezivanje ovih komponenti sa izvorima podataka. 

\subsection{Komponente za navigaciju}

Komponente za navigaciju omogućavaju korisniku da se kreće kroz podatke i različite delove aplikacije. U \textit{Windows}-aplikacijama, tipične komponente za navigaciju uključuju \textit{MenuStrip}, \textit{ToolStrip}, \textit{Toolbar} ili \textit{Navigation Pane}, koje omogućavaju otvaranje različitih obrazaca, filtriranje podataka, prelazak između tabela ili modula aplikacije. Na primer, u programu \textit{Microsoft Access}, \textit{Navigation Pane} prikazuje sve objekte baze podataka (tabele, obrasce, izveštaje) i omogućava brzi pristup njihovom otvaranju i upravljanju. \\

Dodatno, vizuelni elementi kao što su dugmad, kartice (\textit{tabs}) i paneli omogućavaju intuitivno upravljanje sadržajem i prelazak između različitih funkcionalnosti aplikacije, čime se poboljšava korisničko iskustvo i olakšava rad sa podacima. 

\newpage

\textbf{Literatura:}

\vspace{0.5cm}

\begin{itemize}
    \item Petlja – Uvod u baze podataka \\
    Dostupno na: \texttt{https://petlja.org/sr-Latn-RS/kurs/4654/0}
    \item Matematički fakultet u Beogradu – Baze podataka (nastavni materijali) \\
    Dostupno na: \texttt{https://www.bazepodataka.matf.bg.ac.rs/}
    \item S. Malkov – Uvod u relacione baze podataka \\ Dostupno na: \texttt{https://poincare.matf.bg.ac.rs/~smalkov/nastava.urbp.php}
    \item Poslovna informatika – Referencijalni integritet u bazama podataka \\
    Dostupno na: \texttt{https://poslovnainformatika.rs/access/referencijalni-integritet/}
    \item Poslovna informatika – Kreiranje upita u MS Access-u \\
    Dostupno na: \texttt{https://poslovnainformatika.rs/access/kreiranje-upita/?cn-reloaded=1}
    \item Edukacija.rs – SQL naredbe i skupovi podataka \\ Dostupno na: \texttt{https://edukacija.rs/it/baze-podataka/sql-naredbe-setovi}
    \item Petlja – Baze podataka (kompletan kurs) \\
    Dostupno na: \texttt{https://petlja.org/sr-Latn-RS/kurs/4654/}
\end{itemize}



\end{document}
